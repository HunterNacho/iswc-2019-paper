\RequirePackage{amsmath}
\documentclass{llncs}
\usepackage[utf8]{inputenc}
\PassOptionsToPackage{table}{xcolor}
\pagestyle{headings}  % This option shows page numbers, it have to be removed in the final document.
\usepackage{booktabs}
\usepackage{amssymb}
\usepackage{tikz}
\usepackage{scalerel}
\usepackage{comment}
\usepackage{framed}
\usepackage{listings}
%\usepackage{pifont}% http://ctan.org/pkg/pifont
%\usepackage{balance}
%\usepackage{stmaryrd}
\usepackage{adjustbox}
\usepackage{multirow}
\usepackage{enumitem}
\usepackage{xspace}
\usepackage{url}
\usepackage{inconsolata}
\usepackage{etoolbox}

%%%%%%%%%%%%%%%%%%%%%%%%%%%%%%%%%%%%%%%%%%%%%%%%%%%%%%
%% Quickly switch from conference to extended version
\providetoggle{conf}
\settoggle{conf}{false}

% text only to appear in extended version
\newcommand{\ever}[1]{\iftoggle{conf}{}{#1}}

% text only to appear in the conference version
\newcommand{\cver}[1]{\iftoggle{conf}{#1}{}}

% alternate text for both versions
\newcommand{\cever}[2]{\iftoggle{conf}{#1}{#2}}

% environment for long extended-only text
\iftoggle{conf}%
 {\excludecomment{exver}}%
 {\newenvironment{exver}}%
%%%%%%%%%%%%%%%%%%%%%%%%%%%%%%%%%%%%%%%%%%%%%%%%%%%%%% 


\usepackage{floatrow}
% Table float box with bottom caption, box width adjusted to content
\newfloatcommand{capbtabbox}{table}[][\FBwidth]

\newcommand{\ah}[1]{{\color{blue}\textsc{ah:} #1}}

\makeatletter
\renewcommand\paragraph{\@startsection{paragraph}{4}{\z@}%
	{1ex \@plus1ex \@minus.2ex}%
	{-1em}%
	{\normalfont\normalsize\itshape}}

%%%%%%%%%%%%%%%%%%%%%%%%%%%%%%%%%%%%%%%%%%%%%%%%%
% SPARQL Listing style
%%%%%%%%%%%%%%%%%%%%%%%%%%%%%%%%%%%%%%%%%%%%%%%%%
\usepackage{listings}

\colorlet{punct}{red!60!black}
\definecolor{background}{HTML}{EEEEEE}
\definecolor{delim}{RGB}{120,20,40}
\definecolor{keyw}{RGB}{0,0,192}
\colorlet{numb}{magenta!60!black}

\lstdefinelanguage{sparql}{
	sensitive=false,
	extendedchars=true,
	literate={á}{{\'a}}1 {é}{{\'e}}1 {í}{{\'{\i}}}1 {ó}{{\'o}}1 {ú}{{\'u}}1
	{Á}{{\'A}}1 {É}{{\'E}}1 {Í}{{\'I}}1 {Ó}{{\'O}}1 {Ú}{{\'U}}1
	{ü}{{\"u}}1 {Ü}{{\"U}}1 {ñ}{{\~n}}1 {Ñ}{{\~N}}1 {¿}{{?``}}1 {¡}{{!``}}1
	{<}{{{\color{delim}<}}}{1}
	{>}{{{\color{delim}>}}}{1}
	{?}{{{\color{delim}?}}}{1}
	{*}{{{\color{delim}*}}}{1}
	{+}{{{\color{delim}+}}}{1}
	{/}{{{\color{delim}/}}}{1}
	{,}{{{\color{punct}{,}}}}{1}
	{;}{{{\color{punct}{;}}}}{1}
	{.}{{{\color{punct}{.}}}}{1}
	{:}{{{\color{punct}{:}}}}{1}
	{\{}{{{\color{delim}{\{}}}}{1} {\}}{{{\color{delim}{\}}}}}{1},
	morekeywords={ask,select,from,where,order,by,distinct,limit,offset,optional,union,filter,prefix,bound,desc,regex,str,group,not,exists,minus,service,certain,maybe}
}

\lstdefinestyle{sparqld}{
	basicstyle=\scriptsize\ttfamily,
	identifierstyle=\color{black},
	keywordstyle=\color{keyw}\bfseries,
	ndkeywordstyle=\color{greenCode}\bfseries,
	stringstyle=\color{ocherCode}\ttfamily,
	commentstyle=\color{darkgray}\ttfamily,
	language={sparql},
	tabsize=2,
	showtabs=false,
	showspaces=false,
	showstringspaces=false,
	extendedchars=true,
	escapechar=`,
	frame={single},
	breaklines=true,
	basewidth=0.5em,
	moredelim=[is][\color{magenta}]{~}{~},
	moredelim=**[is][\color{gray}]{£}{£},
	moredelim=**[is][\color{blue!50!black}]{$}{$},
	moredelim=[is][\color{orange!80!black}]{!}{!},
	moredelim=**[is][\color{green!50!black}]{¬}{¬},
	xleftmargin=2ex,
	xrightmargin=1ex,
	aboveskip=1.5ex,
	belowskip=1.5ex
}

\makeatletter
\newcommand{\sqbox}{%
	\collectbox{%
		\@tempdima=\dimexpr\width-\totalheight\relax
		\ifdim\@tempdima<\z@
		\fbox{\hbox{\hspace{-.5\@tempdima}\BOXCONTENT\hspace{-.5\@tempdima}}}%
		\else
		\ht\collectedbox=\dimexpr\ht\collectedbox+.5\@tempdima\relax
		\dp\collectedbox=\dimexpr\dp\collectedbox+.5\@tempdima\relax
		\fbox{\BOXCONTENT}%
		\fi
	}%
}
\makeatother
%%%%%%%%%%%%%%%%%%%%%%%%%%%%%%%%%%%%%%%%%%%%%%%%%
% /SPARQL Listing style
%%%%%%%%%%%%%%%%%%%%%%%%%%%%%%%%%%%%%%%%%%%%%%%%%

%%%%%%%%%%%%%%%%%%%%%%%%%%%%%%%%%%%%%%%%%%%%%%%%%
% TiKz for RDF graphs
%%%%%%%%%%%%%%%%%%%%%%%%%%%%%%%%%%%%%%%%%%%%%%%%%

\usetikzlibrary{shapes,arrows,positioning,fit,backgrounds,matrix,chains,scopes,calc}

\newcommand{\hsp}{\vphantom{Ag}}
\tikzset{
	std/.style={ 
		draw,
		circle,
		anchor=center,
		inner sep=0pt,
		minimum size=5pt
	},
	lab/.style={ 
		text centered,
		fill=white, 
		inner sep=0.7pt,
		font=\tt\small\hsp},
	iri/.style={
		draw=black!50!white, 
		rectangle,
		rounded corners,
		thick,
		text centered,
		top color=white, 
		bottom color=black!15,
		%opacity=0.7,
		%text opacity=1,
		font=\tt\small\hsp,
		anchor=center},
	lit/.style={
		draw=black!50!white, 
		rectangle,
		thick,
		text centered,
		top color=white, 
		bottom color=black!15, 
		anchor=center,
		%opacity=0.7,
		%text opacity=1,
		font=\tt\small\hsp},
	arrout/.style={
		->,
		-latex,
		%draw=black!50, 
		%fill=black!50,
		%thick,
		font=\tt\small\hsp},
	arrin/.style={
		<-,
		latex-,
		%draw=black!50, 
		%fill=black!50,
		%thick,
		font=\tt\small\hsp},
	arrinout/.style={
		<->,
		latex-latex,
		%draw=black!50, 
		%fill=black!50,
		%thick,
		font=\tt\small\hsp},
	arrstd/.style={
		-,
		%draw=black!50, 
		%fill=black!50,
		thick},
	dashmed/.style={
		-,
		%draw=black!50, 
		%fill=black!50,
		thick,
		dash pattern=on 3pt off 3pt,
		font=\tt\small\hsp},
	fade/.style={
		opacity=0.4,
		text opacity=0.4
	},
	fadet/.style={
		opacity=1,
		text opacity=0.4
	},
	every loop/.style={
		<-,
		latex-,
		fill=black!50,
		min distance=10mm,
		in=0,
		out=60,
		looseness=10,
		draw=black!50,
		thick,
		font=\tt\small\hsp
	},
	lean/.style={
		dotted,
		blue
	},
	leane/.style={
		dotted,
		blue
	},
	leanl/.style={
		text=blue
	}	
}

\newlength{\hgap}
\newlength{\vgap}
\setlength{\hgap}{2cm}
\setlength{\vgap}{1.2cm}

%%%%%%%%%%%%%%%%%%%%%%%%%%%%%%%%%%%%%%%%%%%%%%%%%
% /TiKz for RDF graphs
%%%%%%%%%%%%%%%%%%%%%%%%%%%%%%%%%%%%%%%%%%%%%%%%%

%%%%%%%%%%%%%%%%%%%%%%%%%%%%%%%%%%%%%%%%%%%%%%%%%
% PGFplots
%%%%%%%%%%%%%%%%%%%%%%%%%%%%%%%%%%%%%%%%%%%%%%%%%
\usepackage{pgfplots}
\usepgfplotslibrary{groupplots}
\usepackage{pgfplotstable}
\pgfplotsset{compat=newest}
\usetikzlibrary{pgfplots.statistics}

\makeatletter
\pgfplotsset{
	boxplot prepared from table/.code={
		\def\tikz@plot@handler{\pgfplotsplothandlerboxplotprepared}%
		\pgfplotsset{
			/pgfplots/boxplot prepared from table/.cd,
			#1,
		}
	},
	/pgfplots/boxplot prepared from table/.cd,
	table/.code={\pgfplotstablecopy{#1}\to\boxplot@datatable},
	row/.initial=0,
	make style readable from table/.style={
		#1/.code={
			\pgfplotstablegetelem{\pgfkeysvalueof{/pgfplots/boxplot prepared from table/row}}{##1}\of\boxplot@datatable
			\pgfplotsset{boxplot/#1/.expand once={\pgfplotsretval}}
		}
	},
	make style readable from table=lower whisker,
	make style readable from table=upper whisker,
	make style readable from table=lower quartile,
	make style readable from table=upper quartile,
	make style readable from table=median,
}
\makeatother

%%%%%%%%%%%%%%%%%%%%%%%%%%%%%%%%%%%%%%%%%%%%%%%%%
% /PGFplots
%%%%%%%%%%%%%%%%%%%%%%%%%%%%%%%%%%%%%%%%%%%%%%%%%

\usepackage{arydshln}

%%%%%%%%%%%%%%%%%%%%%%%%%%%%%%%%%%%%%%%%%%%%%%%%%
% Hyperref
%%%%%%%%%%%%%%%%%%%%%%%%%%%%%%%%%%%%%%%%%%%%%%%%%
\usepackage{hyperref}
\definecolor{dark-blue}{rgb}{0.0,0.0,0.2}
\definecolor{dark-green}{rgb}{0.0,0.2,0.0}
\definecolor{dark-red}{rgb}{0.2,0.0,0.0}
\hypersetup{
	colorlinks, linkcolor={dark-red},
	citecolor={dark-green}, urlcolor={dark-blue},
	pdftitle={Versioned Queries over RDF Archives using Native SPARQL Storage},    % title
	pdfauthor={Ignacio Cuevas, Aidan Hogan},     % author
	pdfsubject={MEPDaW 2020: Managing the Evolution and Preservation of the Data Web},   % subject of the document
	pdfkeywords={sparql;} {versioning;} {rdf archives;} {dynamics}, % list of keywords
}

%%%%%%%%%%%%%%%%%%%%%%%%%%%%%%%%%%%%%%%%%%%%%%%%%
% /Hyperref
%%%%%%%%%%%%%%%%%%%%%%%%%%%%%%%%%%%%%%%%%%%%%%%%%


%%%%%%%% MACROS

\newcommand{\B}{\ensuremath{\mathbf{B}}\xspace}
\newcommand{\I}{\ensuremath{\mathbf{I}}\xspace}
\renewcommand{\L}{\ensuremath{\mathbf{L}}\xspace}
\newcommand{\V}{\ensuremath{\mathbf{V}}\xspace}

\newcommand{\cpx}[2]{\ensuremath{\textsc{#1-}\mathrm{#2}}\xspace}
\newcommand{\npc}{\cpx{NP}{complete}}
\newcommand{\nph}{\cpx{NP}{hard}}
\newcommand{\psc}{\cpx{PSpace}{complete}}
\newcommand{\psh}{\cpx{PSpace}{hard}}
\newcommand{\gic}{\cpx{GI}{complete}}
\newcommand{\gih}{\cpx{GI}{hard}}
\newcommand{\dpc}{\cpx{DP}{complete}}
\newcommand{\dph}{\cpx{DP}{hard}}
\newcommand{\conpc}{\cpx{coNP}{complete}}
\newcommand{\conph}{\cpx{coNP}{hard}}
\newcommand{\ptp}{\ensuremath{\Pi_2^P}}
\newcommand{\ptph}{\cpx{\ptp}{hard}}
\newcommand{\ptpc}{\cpx{\ptp}{complete}}

\newcommand{\tch}[1]{\textbf{#1}}
\newcommand{\rid}[1]{\textsc{#1}}
\newcommand{\ttl}[1]{\textsf{#1}}
\newcommand{\tid}[1]{\textsc{#1}}
%\newcommand{\dom}{\tid{dom}}
%\newcommand{\rng}{\tid{rng}}
%\newcommand{\sC}{\tid{sC}}
%\newcommand{\sP}{\tid{sP}}

\newcommand{\ssyn}[3]{[\ensuremath{#1\,\textsc{#2}\,#3}]}
\newcommand{\sand}[2]{\ssyn{#1}{and}{#2}}
\newcommand{\suni}[2]{\ssyn{#1}{union}{#2}}
\newcommand{\sopt}[2]{\ssyn{#1}{opt}{#2}}
\newcommand{\sfil}[2]{\ensuremath{\textsc{filter}_{#2}(#1)}}
\newcommand{\ssel}[2]{\ensuremath{\textsc{select}_{#2}(#1)}}
\newcommand{\sseld}[2]{\ensuremath{\textsc{select}^\Delta_{#2}(#1)}}
\newcommand{\dom}[1]{\ensuremath{\mathrm{dom}(#1)}}
\newcommand{\can}[1]{\ensuremath{\mathrm{can}(#1)}}
\newcommand{\com}[2]{\ensuremath{#1 \sim #2}}

%\newcommand{\hsc}[1]{{\footnotesize\MakeUppercase{#1}}}
\newcommand{\hsc}[1]{#1}

%\newcommand{\utp}[1]{\textsc{tp}{(#1)}}
\newcommand{\ufo}[1]{\textsf{\hsc{\upshape #1}}}
\newcommand{\uand}[1]{\ensuremath{\ufo{and}(#1)}}
\newcommand{\uuni}[1]{\ensuremath{\ufo{union}(#1)}}
\newcommand{\usel}[2]{\ensuremath{\ufo{select}_{#2}(#1)}}
\newcommand{\useld}[2]{\ensuremath{\ufo{select}^\Delta_{#2}(#1)}}

\newcommand{\uandn}{\ensuremath{\ufo{and}}}
\newcommand{\uunin}{\ensuremath{\ufo{union}}}

\newcommand{\bn}[1]{\texttt{\_:#1}}
\newcommand{\iri}[1]{\texttt{:#1}}
\newcommand{\var}[1]{\texttt{?#1}}

\newcommand{\ican}[1]{\ensuremath{\textsc{iCan}(#1)}}
\newcommand{\ecan}[1]{\ensuremath{\textsc{eCan}(#1)}}

\newcommand{\ev}[2]{\ensuremath{#1(#2)}}

\def\ojoin{\setbox0=\hbox{$\bowtie$}%
	\rule[0.18ex]{.25em}{.5pt}\llap{\rule[.9ex]{.25em}{.5pt}}}
\def\loj{\mathbin{\ojoin\mkern-5.8mu\bowtie}}

\newcommand{\da}{\ensuremath{:\nolinebreak\mkern-1.2mu\nolinebreak=}}

\newcommand{\qedr}{\begin{flushright}\qed\end{flushright}}

\newcommand{\yt}{\ding{51}}%
\newcommand{\nt}{\ding{55}}%

\newcommand{\para}[1]{\smallskip\noindent\textbf{#1:}}

\newcommand{\mq}{\textsc{mq}\xspace}
\newcommand{\mqs}{\textsc{mq}s\xspace}
\newcommand{\ucq}{\textsc{ucq}\xspace}
\newcommand{\ucqs}{\textsc{ucq}s\xspace}
\newcommand{\cq}{\textsc{cq}\xspace}
\newcommand{\cqs}{\textsc{cq}s\xspace}
\newcommand{\cuq}{\textsc{ucq}\xspace}
\newcommand{\cuqs}{\textsc{ucq}s\xspace}

%%%%%%%%

\graphicspath{ {images/} }	

\begin{document}
\title{Versioned Queries over RDF Archives using Native SPARQL Storage}

\author{Ignacio Cuevas \and Aidan Hogan}
\institute{Department of Computer Science, University of Chile \& IMFD Chile}

\maketitle

\begin{abstract}
Although many prominent Linked Datasets are in a state of constant evolution, historical data are rarely published in a standard way. Though a number of works have recently addressed the issue of querying and managing archives of versioned RDF data, most propose dedicated solutions, which incurs a high deployment cost. We rather propose solutions for representing archives of versioned RDF data using the standard SPARQL language, allowing existing implementations to be used. For querying, we consider two versioning primitives: viewing results from a particular version, or viewing changes between two consecutive versions. We identify and compare several models by which historical versions of an RDF graph can be stored as named graphs, and discuss how base queries can be rewritten in each model per the two aforementioned primitives. We then present a performance comparison of these models and baseline (non-)versioned queries with respect to 23 weekly versions of Wikidata and 232 user-defined queries, using Virtuoso as a reference SPARQL implementation.
\end{abstract}


\section{Introduction}

A key aspect of the Web is its dynamic nature, where documents are frequently updated, deleted and added. Likewise when we speak of the Semantic Web, it is important to consider that sources may be dynamic and RDF datasets are subject to change. The traditional Web and the Semantic Web thus share similar strengths and limitations regarding the dynamics of information. In terms of strengths, the flexible nature of both webs means that documents can be updated with little restriction and with little need for centralised coordination. On the other hand, in terms of limitations, neither web has built in support for preservation, nor for versioning; for example, on the traditional Web, the HTTP protocol does not permit requesting a past version of a webpage. For this reason, a number of specialised ``Web archives'' have emerged that attempt to capture and track different intermittent versions of documents on the Web, the most prominent of which is the Internet Archive~\cite{JaffeK09}. While similar techniques can be applied to Semantic Web documents containing RDF data -- simply archiving and thus preserving the syntax of the document itself -- the structured nature of such content means that an RDF archive can potentially do a lot more.

A number of works have looked at archiving RDF content from the Web. One such example is the Dynamic Linked Data Observatory (DyLDO)~\cite{KaferAUOH13}, which has been tracking and archiving changes in a sample of around 100,000 RDF documents each week since 2013. Other works rather focus on aspects relating to detecting change~\cite{TummarelloMBE07,ZeginisTC11,PapavasileiouFFKC13,KaferAUOH13,DividinoKG14,RoussakisCSFS15,NishiokaS18}, notifying of changes~\cite{TummarelloMBE07,PopitschH11,PassantM10,TrampFEA10}, versioning data~\cite{VolkelG06,GraubeHU14,KhuranaD16}, and so forth. 

% RDF archives~\cite{FernandezUPK19}



Currently, a large number of existing datasets are based on RDF due to its extense popularity, Wikidata being one the most notorious. Wikidata~\cite{VrandecicK14} is a massive dataset, with over 42 million items~\footnote{\url{https://www.wikidata.org/wiki/Wikidata:Statistics}}, which is based on all the available data of Wikipedia and structured on RDF.


Wikidata is in constant change, due to it being an open dataset that allows being edited by users. Furthermore, data is being added often, which makes Wikidata very dynamic in some areas. Given those circumstances, it can be said that Wikidata has many ``versions'' over time, where different data is available at different time intervals, making it possible to analyze its time component.


Dataset versioning in RDF/SPARQL is a research topic yet to be fully explored. While there are a number of implementations, most of them rely on specialized indices or SPARQL extensions. As such, it would be relevant to study the possibility of developing a versioning system and methodology for queries using only base SPARQL.


Such system could be used as version control for the dataset; administrators would be able to detect erroneous or malicious editions on data that doesn't normally change. On top of that, it would be possible to analyze the evolution of data, possibly predicting future changes. The system could also be used on datasets other than Wikidata, where data history may be poorly (or not at all) preserved.

\section{Related Work}
Gutierrez et al.~\cite{GutierrezHV07} describe a framework that allows time representation in RDF using annotations. An important distinction to make is that their work is focused on \textbf{labeling} over \textbf{versioning}; \textbf{labeling} means adding meta-data to query result validity, meanwhile \textbf{versioning} implies simultaneously keeping several versions of the same dataset. This work also formally defines labels and time intervals, but does not cover an implementation for either.


Zimmerman et al.~\cite{DBLP:journals/ws/ZimmermannLPS12} extend RDF, allowing the use of generalized annotations, which can be used to represent time intervals and associated to each triple to represent their validity. Same as the previously mentioned work, the objectives seeked by Zimmerman et al. do not completely align with the present work, due to the fact that an RDF extension is used. It will, however, be considered to compare results.


Grandi~\cite{Grandi10} proposes an extension for SPARQL, including a time component in queries. No implementation details are provided, however. The methodology used for building timed queries will be compared to the one this work defines.


The prior three works share their focus on formal definitions over implementations. They also do not consider efficiency or scalability analysis and employ SPARQL extensions and/or specialized indices. In light of these facts, said works will be considered only to compare results for the most part.


Lastly, Tappolet and Bernstein~\cite{TappoletB09} add a time component to RDF's syntax, proposing an eficient method to make SPARQL queries on it, as well. The dataset is annotated with time intervals corresponding to each triple's validity, and a specialized index compliments the query engine. Since this is an alternative solution to the explored challenge, it will be used to compare results, while also evaluating whether their specialized index is necessary and if similar results can be achieved using standar SPARQL.
\section{Preliminaries} % Maybe not
\section{(Method)} % Name of our method goes here
\section{Evaluation} % This two sections can
\section{Results}    % be merged for better flow
\section{Conclusion}
\newpage
\bibliographystyle{unsrt}
\bibliography{paper}
\end{document}